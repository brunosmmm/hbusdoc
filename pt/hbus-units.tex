A especificação HBUS, através do uso de objetos invisíveis e utilizando a sintaxe da seção \ref{sec:hiddenobj}, através do campo UNIT, implementa algumas unidades padrão do SI, para melhor visualização dos dados disponibilizados pelos dispositivos.

\section{Unidades de medidas}

As unidades suportadas pelo sistema são um conjunto de reduzido de unidades comuns utilizadas no SI.

\begin{table}[H]
\centering
\caption{unidades padrão no HBUS}
\begin{tabular}{l l l}
\hline
Unidade		&	Símbolo SI	&	String HBUS\\
\hline
\textit{volt} & V			&	V\\
\textit{ampère} & A			&   A\\
\textit{watt}   & W			&   W\\
\textit{pascal} & Pa			&   Pa\\
\textit{graus Celsius} & $^{\circ} C$ & C\\
\textit{ohm}		& $\Omega$	&	R\\
\end{tabular}
\end{table}

\section{Prefixos suportados}

Vários prefixos são suportados para uso conjunto com as unidades. Estes prefixos devem vir antes da unidade, sem espaços.

A tabela abaixo lista os prefixos suportados e o caractere para uso nos dispositivos.

\begin{table}[H]
\centering
\caption{prefixos suportados no HBUS}
\begin{tabular}{l c c l}
\hline
Prefixo		&	Símbolo SI	&	Caractere HBUS	&	Potência\\
\hline
%pico			&	p			&	p				&	$10^{-12}$\\
%nano			&	n			&	n				&	$10^{-9}$\\
micro		&	$\mu$		&	u				&	$10^{-6}$\\
mili			&	m			&	m				&	$10^{-3}$\\
kilo			&	k			&	k				&	$10^3$\\
mega			&	M			&	M				&	$10^6$\\
\hline
\end{tabular}
\end{table}
