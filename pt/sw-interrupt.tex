
Neste capítulo é exposta a definição das operações de interrupção por dispositivos escravos no barramento HBUS.

\section{Necessidade de interrupções}

A necessidade da existência de um mecanismo de interrupções no sentido escravos -> mestres é bem evidente em qualquer sistema computacional que envolve comunicação num barramento que contém um mestre e vários escravos, que dividem o canal de comunicação.

Em especial, o barramento HBUS, tendo como princípio a minimização do uso do canal de comunicação, atingida através da restrição de início de tráfego no canal por parte do mestre, necessita de um mecanismo que possibilite os escravos a iniciar uma comunicação com o mestre em uma ocasião de extrema necessidade ou exceções encontradas.

\section{O processo de interrupção}

Utiliza-se o sinal \textbf{FREEBUS} alocado no barramento exclusivamente para as operações de interrupção.

O processo de interrupção ocorre na sequência exibida a seguir:

\begin{enumerate}

\item O escravo leva o sinal FREEBUS ao valor lógico 0. Isto sinaliza aos demais escravos que as transmissões devem terminar o mais rápido possível.

\item O escravo escuta o canal de comunicação e assim que estiver livre, emite o comando de interrupção no barramento INT, identificando-se no processo.

\item O mestre emite um comando BUSLOCK, direcionado ao escravo emissor da interrupção. O barramento é travado exclusivamente para comunicação do mestre com o escravo.

\item Ao confirmar o travamento do barramento, o escravo libera o sinal FREEBUS, colocando-o em alta impedância.

\item São transferidas informações específicas à interrupção e suas consequências.

\item Ao fim das transmissões específicas, o mestre emite o comando BUSUNLOCK e libera o barramento para outras comunicações.


\end{enumerate}

Não é requerido que as interrupções sejam detectadas por hardware exclusivo. A implementação utilizada como padrão é a detecção da interrupção por software.

\section{Autenticação de interrupções}

As interrupções são eventos muitas vezes urgentes e importantes que provavelmente deflagrarão ações pré-programadas. É muito interessante que esse mecanismo também tenha possibilidade de ser seguro.

Surge então o conceito de autenticação de interrupções, que é uma funcionalidade futura do HBUS. Este sistema envolve o processo de interrupção já descrito e o sistema de autenticação de escravo.
